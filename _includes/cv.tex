%============================================================================%
%
%	DOCUMENT DEFINITION
%
%============================================================================%

\documentclass[10pt,A4,english]{article}


%----------------------------------------------------------------------------------------
%	ENCODING
%----------------------------------------------------------------------------------------

% we use utf8 since we want to build from any machine
\usepackage[utf8]{inputenc}
\usepackage[USenglish]{isodate}
\usepackage{fancyhdr}
% \usepackage[numbers]{natbib}

%----------------------------------------------------------------------------------------
%	LOGIC
%----------------------------------------------------------------------------------------

% provides \isempty test
\usepackage{xstring, xifthen}
\usepackage{enumitem}
\usepackage[english]{babel}
\usepackage{blindtext}
\usepackage{pdfpages}
\usepackage{changepage}
%----------------------------------------------------------------------------------------
%	FONT BASICS
%----------------------------------------------------------------------------------------

% some tex-live fonts - choose your own
\usepackage[default]{raleway}

% set font default
\renewcommand*\familydefault{\sfdefault}
\usepackage[T1]{fontenc}

% more font size definitions
\usepackage{moresize}

%----------------------------------------------------------------------------------------
%	FONT AWESOME ICONS
%----------------------------------------------------------------------------------------

% include the fontawesome icon set
\usepackage{fontawesome5}

% use to vertically center content
% credits to: http://tex.stackexchange.com/questions/7219/how-to-vertically-center-two-images-next-to-each-other
\newcommand{\vcenteredinclude}[1]{\begingroup
\setbox0=\hbox{\includegraphics{#1}}%
\parbox{\wd0}{\box0}\endgroup}
\newcommand{\tab}[1]{\hspace{.2\textwidth}\rlap{#1}}
% use to vertically center content
% credits to: http://tex.stackexchange.com/questions/7219/how-to-vertically-center-two-images-next-to-each-other
\newcommand*{\vcenteredhbox}[1]{\begingroup
\setbox0=\hbox{#1}\parbox{\wd0}{\box0}\endgroup}

% icon shortcut
\newcommand{\icon}[2] {
	\makebox(#2, #2){\textcolor{maincol}{\textcolor{maincol}{\faIcon{#1}}}}
}


% icon with text shortcut
\newcommand{\icontext}[3]{
	\vcenteredhbox{\icon{#1}{#2}}  \hspace{2pt}  \parbox{0.9\mpwidth}{\textcolor{black}{#3}}
}

% icon with website url
\newcommand{\iconhref}[4]{
    \vcenteredhbox{\icon{#1}{#2}}  \hspace{2pt} \href{#4}{\textcolor{black}{#3}}
}

% icon with email link
\newcommand{\iconemail}[3]{
    \vcenteredhbox{\icon{#1}{#2}}  \hspace{2pt} \href{mailto:#3}{\textcolor{black}{#3}}
}

%----------------------------------------------------------------------------------------
%	PAGE LAYOUT  DEFINITIONS
%----------------------------------------------------------------------------------------

% page outer frames (debug-only)
% \usepackage{showframe}

% we use paracol to display breakable two columns
\usepackage{paracol}
\usepackage{tikzpagenodes}
\usetikzlibrary{calc}
\usepackage{lmodern}
\usepackage{multicol}
\usepackage{lipsum}
\usepackage{atbegshi}
% define page styles using geometry
\usepackage[a4paper]{geometry}

% remove all possible margins
\geometry{top=1cm, bottom=1cm, left=1cm, right=1cm}

\usepackage{fancyhdr}
\pagestyle{empty}

% space between header and content
% \setlength{\headheight}{0pt}

% indentation is zero
\setlength{\parindent}{0mm}

%----------------------------------------------------------------------------------------
%	TABLE /ARRAY DEFINITIONS
%----------------------------------------------------------------------------------------

% extended aligning of tabular cells
\usepackage{array}

% custom column right-align with fixed width
% use like p{size} but via x{size}
\newcolumntype{x}[1]{%
>{\raggedleft\hspace{0pt}}p{#1}}%


%----------------------------------------------------------------------------------------
%	GRAPHICS DEFINITIONS
%----------------------------------------------------------------------------------------

%for header image
\usepackage{graphicx}

% use this for floating figures
% \usepackage{wrapfig}
% \usepackage{float}
% \floatstyle{boxed}
% \restylefloat{figure}

%for drawing graphics
\usepackage{tikz}
\usepackage{ragged2e}
\usetikzlibrary{shapes, backgrounds,mindmap, trees}

%----------------------------------------------------------------------------------------
%	Bibliography
%----------------------------------------------------------------------------------------

%----------------------------------------------------------------------------------------
%	Color DEFINITIONS
%----------------------------------------------------------------------------------------
\usepackage{transparent}
\usepackage{color}

% primary color
\definecolor{maincol}{HTML}{A56DAA}

% accent color, secondary
% \definecolor{accentcol}{RGB}{ 250, 150, 10 }

% dark color
\definecolor{darkcol}{RGB}{ 70, 70, 70 }

% light color
\definecolor{lightcol}{RGB}{245,245,245}

\definecolor{accentcol}{HTML}{6A2C70}



% Package for links, must be the last package used
\usepackage[hidelinks]{hyperref}

% returns minipage width minus two times \fboxsep
% to keep padding included in width calculations
% can also be used for other boxes / environments
\newcommand{\mpwidth}{\linewidth-\fboxsep-\fboxsep}



%============================================================================%
%
%	CV COMMANDS
%
%============================================================================%

%----------------------------------------------------------------------------------------
%	 CV TEXT
%----------------------------------------------------------------------------------------

% base class to wrap any text based stuff here. Renders like a paragraph.
% Allows complex commands to be passed, too.
% param 1: *any
\newcommand{\cvtext}[1] {
	\begin{tabular*}{1\mpwidth}{p{0.98\mpwidth}}
		\parbox{1\mpwidth}{#1}
	\end{tabular*}
}
\newcommand{\cvtextsmall}[1] {
	\begin{tabular*}{0.8\mpwidth}{p{0.8\mpwidth}}
		\parbox{0.8\mpwidth}{#1}
	\end{tabular*}
}
%----------------------------------------------------------------------------------------
%	CV SECTION
%----------------------------------------------------------------------------------------

% Renders a a CV section headline with a nice underline in main color.
% param 1: section title
\newlength{\barw}
\newcommand{\cvsection}[1] {
	\vspace{14pt}
	\settowidth{\barw}{\textbf{\LARGE #1}}
	\cvtext{
		\textbf{\LARGE{\textcolor{darkcol}{#1}}}\\[-4pt]
		\textcolor{accentcol}{ \rule{\barw}{1.5pt} } \\
	}
}

\newcommand{\cvsubsection}[1] {
	\vspace{14pt}
	\settowidth{\barw}{\textbf{\Large #1}}
	\cvtext{
		\textbf{\Large{\textcolor{darkcol}{#1}}}\\[-4pt]
		\textcolor{accentcol}{ \rule{\barw}{1.5pt} } \\
	}
}

\newcommand{\cvheadline}[1] {
	\vspace{16pt}
	\cvtext{
		\textbf{\Huge{\textcolor{accentcol}{#1}}}\\[-4pt]

	}
}

\newcommand{\cvsubheadline}[1] {
	\vspace{16pt}
	\cvtext{
		\textbf{\huge{\textcolor{darkcol}{#1}}}\\[-4pt]

	}
}
%----------------------------------------------------------------------------------------
%	META SKILL
%----------------------------------------------------------------------------------------

% Renders a progress-bar to indicate a certain skill in percent.
% param 1: name of the skill / tech / etc.
% param 2: level (for example in years)
% param 3: percent, values range from 0 to 1
\newcommand{\cvskill}[3] {
	\begin{tabular*}{1\mpwidth}{p{0.72\mpwidth}  r}
 		\textcolor{black}{\textbf{#1}} & \textcolor{maincol}{#2}\\
	\end{tabular*}%

	\hspace{4pt}
	\begin{tikzpicture}[scale=1,rounded corners=2pt,very thin]
		\fill [lightcol] (0,0) rectangle (1\mpwidth, 0.15);
		\fill [accentcol] (0,0) rectangle (#3\mpwidth, 0.15);
  	\end{tikzpicture}%
}


%----------------------------------------------------------------------------------------
%	 CV EVENT
%----------------------------------------------------------------------------------------

% Renders a table and a paragraph (cvtext) wrapped in a parbox (to ensure minimum content
% is glued together when a pagebreak appears).
% Additional Information can be passed in text or list form (or other environments).
% the work you did
% param 1: time-frame i.e. Sep 14 - Jan 15 etc.
% param 2:	 event name (job position etc.)
% param 3: Customer, Employer, Industry
% param 4: Short description
% param 5: work done (optional)
% param 6: technologies include (optional)
% param 7: achievements (optional)
\newcommand{\cvevent}[4] {

	% we wrap this part in a parbox, so title and description are not separated on a pagebreak
	% if you need more control on page breaks, remove the parbox
	\parbox{\mpwidth}{
		\begin{tabular*}{1\mpwidth}{p{0.66\mpwidth}  r}
	 		\textcolor{black}{\textbf{#2}} & \colorbox{accentcol}{\makebox[0.32\mpwidth]{\textcolor{white}{\textbf{#1}}}} \\
			\textcolor{maincol}{#3} & \\
		\end{tabular*}\\[8pt]

		\ifthenelse{\isempty{#4}}{}{
			\cvtext{#4}\\
		}
	}
	\vspace{14pt}
}

\newcommand{\cvproj}[4] {
	\parbox{\mpwidth}{
		\begin{tabular*}{1\mpwidth}{p{0.66\mpwidth}  r}
	 		\textcolor{black}{\textbf{#1}} & \\
			\textcolor{maincol}{#2} & \\
		\end{tabular*}\\[4pt]

		\ifthenelse{\isempty{#3}}{}{
			\cvtext{#3}\\
		}

		\ifthenelse{\isempty{#4}}{}{
			\begin{itemize}
				\expandafter\CVprojItems\expandafter{#4}
			\end{itemize}
		}
	}
	\vspace{14pt}
}

% Helper macro to process ||| separated items
\def\CVprojItems#1{%
	\def\ProcessItem##1|||{%
		\ifx\relax##1\relax\else
			\item ##1%
			\expandafter\ProcessItem
		\fi
	}%
	\expandafter\ProcessItem#1|||\relax|||%
}

%----------------------------------------------------------------------------------------
%	 CV META EVENT
%----------------------------------------------------------------------------------------

% Renders a CV event on the sidebar
% param 1: title
% param 2: subtitle (optional)
% param 3: customer, employer, etc,. (optional)
% param 4: info text (optional)
\newcommand{\cvmetaevent}[4] {
	\textcolor{maincol} { \cvtext{\textbf{\begin{flushleft}#1\end{flushleft}}}}

	\ifthenelse{\isempty{#2}}{}{
	\textcolor{black} {\cvtext{\textbf{#2}} }
	}

	\ifthenelse{\isempty{#3}}{}{
		\cvtext{{ \textcolor{maincol} {#3} }}\\
	}

	\cvtext{#4}\\[14pt]
}


% HEADER AND FOOOTER
%====================================
\newcommand\Header[1]{%
\begin{tikzpicture}[remember picture,overlay]
\fill[accentcol]
  (current page.north west) -- (current page.north east) --
  ([yshift=50pt]current page.north east|-current page text area.north east) --
  ([yshift=50pt,xshift=-3cm]current page.north|-current page text area.north) --
  ([yshift=10pt,xshift=-5cm]current page.north|-current page text area.north) --
  ([yshift=10pt]current page.north west|-current page text area.north west) -- cycle;
\node[font=\sffamily\bfseries\color{white},anchor=west,
  xshift=0.7cm,yshift=-0.32cm] at (current page.north west)
  {\fontsize{12}{12}\selectfont {#1}};
\end{tikzpicture}%
}

\newcommand\Footer[1]{%
\begin{tikzpicture}[remember picture,overlay]
\fill[lightcol]
  (current page.south east) -- (current page.south west) --
  ([yshift=-80pt]current page.south east|-current page text area.south east) --
  ([yshift=-80pt,xshift=-6cm]current page.south|-current page text area.south) --
  ([xshift=-2.5cm,yshift=-10pt]current page.south|-current page text area.south) --
  ([yshift=-10pt]current page.south east|-current page text area.south east) -- cycle;
\node[yshift=0.32cm,xshift=9cm] at (current page.south) {\fontsize{10}{10}\selectfont \textbf{\thepage}};
\end{tikzpicture}%
}


%=====================================
%============================================================================%
%
%
%
%	DOCUMENT CONTENT
%
%
%
%============================================================================%
\begin{document}

	\columnratio{0.31}
	\setlength{\columnsep}{2.2em}
	\setlength{\columnseprule}{4pt}
	\colseprulecolor{white}


	% LEBENSLAUF HIERE
	\AtBeginShipoutFirst{\Header{CV}\Footer{1}}
	\AtBeginShipout{\AtBeginShipoutAddToBox{\Header{CV}\Footer{2}}}

	\newpage

	\colseprulecolor{lightcol}
	\columnratio{0.31}
	\setlength{\columnsep}{2.2em}
	\setlength{\columnseprule}{4pt}
		\begin{paracol}{2}
			\begin{leftcolumn}
				%---------------------------------------------------------------------------------------
				%	META IMAGE
				%----------------------------------------------------------------------------------------
				\fcolorbox{white}{white}{\begin{minipage}[c][1.5cm][c]{1\mpwidth}
					\LARGE{\textbf{\textcolor{black}{Connor Aird}}} \\[2pt]
					\normalsize{ \textcolor{black} {Research Software Engineer at UCL} }
				\end{minipage}}

				%---------------------------------------------------------------------------------------
				%	META SKILLS
				%----------------------------------------------------------------------------------------
				\icontext{caret-right}{12}{London, United Kingdom}\\[6pt]
				\icontext{caret-right}{12}{British}\\[6pt]

				\cvsection{Languages}

					\cvskill{C++}{}{0.8} \\[10pt]
					\cvskill{Fortran}{}{0.8} \\[10pt]
					\cvskill{JavaScript}{}{0.2} \\[10pt]
					\cvskill{Python}{}{1} \\[10pt]
					\cvskill{MPI}{}{0.5} \\[10pt]
					\cvskill{OpenMP}{}{0.5} \\[10pt]

				\cvsection{Education}

					\cvmetaevent{09/2016 - 06/2020}{MPhys Physics}{University of York}{1st Class Honours (with distinction)}

				\cvsection{Contact}

					\icontext{map-marker}{16}{London, United Kingdom}\\[6pt]
					\iconemail{envelope}{16}{connoraird@gmail.com}\\[6pt]
					\iconhref{home}{16}{connoraird.github.io}{https://connoraird.github.io}\\[6pt]
					\iconhref{github}{16}{connoraird}{https://github.com/connoraird}\\[6pt]
					\iconhref{linkedin}{16}{connoraird}{https://linkedin.com/in/connoraird}\\[6pt]

			\end{leftcolumn}

			\begin{rightcolumn}
				% \cvsection{Summary}\label{summary}

				% 	Lorem ipsum dolor sit amet, consetetur sadipscing elitr, sed diam nonumy
				% 	eirmod tempor invidunt ut labore et dolore magna aliquyam erat, sed diam
				% 	voluptua. At vero eos et accusam et justo duo dolores et ea rebum. Stet
				% 	clita kasd gubergren, no sea takimata sanctus est Lorem ipsum dolor sit
				% 	amet. Lorem ipsum dolor sit amet, consetetur sadipscing elitr, sed diam
				% 	nonumy eirmod tempor invidunt ut labore et dolore magna aliquyam erat,
				% 	sed diam voluptua. At vero eos et accusam et justo duo dolores et ea
				% 	rebum. Stet clita kasd gubergren, no sea takimata sanctus est Lorem
				% 	ipsum dolor sit amet.

				\cvsection{Experience}\label{experience}

					\cvevent{08/2024 - Present}
						{Research Software Engineer}
						{Centre for Advanced Research Computing \newline University College London}
						{In addition to the responsibilities of an Assistant RSE, I am now responsible for the management of a project, its staffing and the collaboration with the primary investigator. I have also gone beyond helping at software carpentries to now teaching and developing course material.}

					\cvevent{06/2023 - 08/2024}
						{Assitant Research Software Engineer}
						{Centre for Advanced Research Computing \newline University College London}
						{As an Assistant RSE, I collaborated with research colleagues from across UCL to construct, improve and maintain research codes. I provided consulting on software best practices, techniques and designs to help build well tested maintainable software. I also aided in teaching activities through demonstrating software carpentries sessions and aiding with marking.}

					\cvevent{08/2020 - 06/2023}
						{Consultant}
						{Infinity Works \newline Leeds, UK}
						{Whilst at IW, I was part of multiple different teams working on a range of client accounts. My work consisted of developing and maintaining well-tested web applications. I was also responsible for developing cloud infrastructure-as-code for these applications using technologies such as Terraform. Deployments to this infrastructure were managed by myself and the rest of the team through the maintenance of CI/CD pipelines using Jenkins, GitHub actions and other CI/CD tools. Throughout my time at IW, I utilised agile methodologies to effectively deliver work.}

				\cvsection{Projects}\label{projects}

					\cvproj{GLASS, Porting to the python array API}{RSE}{GLASS, Generator for Large Scale Structure, is a python library for performing simulations useful in the field of cosmology.}{Porting existing functions utilising NumPy to be array api compatible.|||Creating a benchmark suite using pytest benchmark.}

					\cvproj{k-Plan Treatment Planning Module}{RSE, Project lead}{K-Plan is a Microsoft native User Interface written in c++ for simulating ultrasound treatment plans. In my role I:}{Schedule staffing on project.|||Prioritse work.|||Implement feature requests, bug fixes and general maintenance.|||Advise on testing strategies.}

					\cvproj{MUSIC, DiRAC GPU feasibility study}{RSE}{MUSIC is a Fortran code which simulates the interiors of stars. This study aimed to determine the potential gains and likely cost of implementing GPU offloading within the code base. In my role I:}{Enabled the OpenMP and CUDA backends of Trilinos (an external library utilised by MUSIC)|||Produced scripts to automate the building of Trilinos, music and Kokkos-tools.|||Wrote submission scripts to allow the rapid production of benchmark data.|||Profiled MUSIC using the NVIDIA NSIGHT system.}

					\cvproj{CONQUEST, eCSE08 Improving Multi-threaded Scaling.}{RSE}{CONQUEST is a large-scale density functional theory (DFT) code written in Fortran. DFT employs principles of quantum physics and is used within many fields, such as physics, chemistry and materials science. In my role I:}{Refactored a particular kernel to maximise serial performance.|||Implemented both OpenMP and BLAS parallelisation within this refactored kernel.|||Carried out performance analysis using benchmarks and profilers to tune this parallel implementation.}
			\end{rightcolumn}
		\end{paracol}


\end{document}
